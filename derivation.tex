\documentclass[11pt]{article}

    \usepackage{ae,lmodern}
    \usepackage[french]{babel}
    \usepackage[utf8]{inputenc}
    \usepackage[T1]{fontenc}
    \usepackage{amsmath}
    \usepackage{amsfonts}
    \usepackage[a4paper, total={16.5cm, 26.5cm}]{geometry}

    \newcommand{\R}{\mathbb R}
    \newcommand{\de}[1]{(#1)'}
    
    \selectlanguage{french}
    \title{Dérivation Terminale S\\
        \large Rappels et Exercices}
    \date{}
    \author{}

\begin{document}
    
    \maketitle
    
    \section{Rappels}
        \subsection{Théorie :}
        On dit qu'une fonction $F$ sur $\R$ est dérivable en un point $x_0$ si
        \begin{equation}
            \lim\limits_{h \to 0} \frac{f(x_0 + h)-f(x_0)}{h} \text{ existe. On nomme alors ce nombre $f'(x_0)$} 
        \end{equation}
        ou (ce qui est équivalent)
        \begin{equation*}
            \lim\limits_{x \to x_0} \frac{f(x)-f(x_0)}{x-x_0} = f'(x_0)            
        \end{equation*}

        Souvenez vous de cette formule car elle permet de prouver certaines limites ($\lim\limits_{x \to 0} \frac{\sin{x}}{x}$ par exemple, essayez !) ou de comprendre la relation entre la dérivée d'une fonction en un point et le coefficiant de la tangeante en ce point.

        \subsection{Formules :}
        \paragraph{formules à connaître impérativement !}
        Elles sont au nombre de 4 : Soient $f$ et $g$ deux fonctions dérivables et $x,y \in  \R $
        \begin{align}
            (a(x) + g(x))' &= a' +g' \label{addition} \\
            (a(x) \times g(x))' &= a'(x)g(x) + a(x)g'(x) \label{produit}\\
            (a(g(x)))' &= a'(g(x))\times g'(x) \label{compose}\\
            (x^y)' &= y\times x^{y-1} \label{exposant}
        \end{align}
        Avec ces formules, vous pouvez retrouver toutes les autres !
        \subparagraph{Exemple} On va retrouver la formule de $\frac{1}{f(x)}$ avec seulement (\ref{compose}) et (\ref{exposant}) :
        \begin{align*}
            \frac{1}{f(x)} &= f(x)^{-1} \tag*{Or, selon (\ref{exposant}),}\\ 
            (y^{-1})' &= -1\times y^{-2} =-\frac{1}{y^2} \tag*{On a déjà la formule pour $1/y$ !}\\
            \text{Donc } \left(\frac{1}{f(x)}\right)' &=- \frac{f'(x)}{f(x)^2} \tag*{Selon (\ref{compose}) en prenant $a(x)=x^{-1}$ et $g(x)=f(x)$} 
        \end{align*}

        \paragraph{formule utile}
        \begin{equation*}
            (\frac{f(x)}{g(x)})' = \frac{f'(x)g(x) - f(x)g'(x)}{g(x)^2}
        \end{equation*}
        Il est bien utile (bien que non nécessaire) de connaître cette formule par coeur.\\
        Exercice : retrouvez-la avec les formules précédentes.
        \subsection{dérivées classiques :}
        \begin{align*}
            \de{cos(x)} &= -sin(x) &\de{sin(x)} &= cos(x) & \de{e^x} &= e^x \\
            \de{ln(x)} &= \frac{1}{x} 
        \end{align*}
        \subparagraph{Remarque 1} vous trouvez peut-être qu'il manque quelques fonctions à cette section comme $\frac{1}{x}$ ou $\sqrt{x}$. Pourtant celles-ci peuvent (et doivent !) être retrouvées via les relations (2), (3) et (4). Comme toujours en maths, nous recherchons l'effort minimum.\\
        Il est inutile de préciser qu'il est vivement conseillé de faire les calculs par soi-même (pensez que $\sqrt{x} = x^{\frac{1}{2}}$ (pourquoi ?)).
        \subparagraph{Remarque 2} Prouvez que, pour $a$ une constante et $f$ une fonction sur $\R$ que
        \begin{equation*}
         \de{a\times f(x)} = a\times f'(x) 
        \end{equation*}
         une bonne fois pour toute et souvenez-vous-en ! Il est inutile est bien trop lent d'appliquer à chaque fois la formule (2) du produit !  

    \section{Exercices}


\end{document}